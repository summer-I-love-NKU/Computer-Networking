\documentclass[lang=cn,11pt]{elegantpaper}
%放在最前面!!!

% ---------设置图片
\usepackage{graphicx}% 插入图片
\usepackage{subfigure}% subfigure——用来插入插入并列分布的图片% caption2——修改图注相关,这里好像不需要
\usepackage{float}% float——确定图片是否为浮动,而不是在一个固定的地方。
\renewcommand{\figurename}{Fig.}
% \renewcommand{\captionlabeldelim}{ }%——这是修改分割符冒号为空格
% ---------设置代码块
\usepackage{listings}
% \usepackage{minted}%代码块高亮显示!!!
% ---------------------------------封面-------------------------------
\title{实验3-3~~~实现拥塞控制算法}
\author{{1813800~~~~沈\ 哲}}
% \institute{{南开大学~~计算机学院,天津~~300000}}
\date{}% \version{0.07}
% % 如果想修改参考文献样式,请把这行注释掉% \usepackage[authoryear]{gbt7714}  % 国标
% % -------------------------------------------摘要和目录----------------------------------------
\begin{document}
    \maketitle
    % \newpage
    \tableofcontents
    % \newpage
% -----------------------------------------------------------------------------------------
% ----------------------------------正文---------------------------------------------------
% -----------------------------------------------------------------------------------------
\section{实验要求}
        利用数据报套接字在用户空间实现面向连接的可靠数据传输,功能包括:建立连接、差错检测、确认重传。
        在任务 3-2 的基础上,选择实现一种拥塞控制算法,也可以是改进的算法。
        要求实现单向传输。对于每一个任务要求给出详细的协议设计。完成给定测试文件的传输,显示传输时间和平均吞吐率。

\section{功能实现}
    \subsection{协议设计}
        UDP是传输层中面向无连接的协议,在编程上服务端和客户端是没有区别的,本实验实现从客户端(发送端)到服务端(接收端)的传输。
        本实验参考了TCP协议来设计数据报相应字段,数据报分为两部分——头部和数据部分。

        实验3-3的协议和3-2完全相同,接近TCP,数据报的结构更容易处理(使用结构体而不是字符数组,虽然字符数组更简单更易于传送,但编程操作上可能更加困难)。
 结构体PACKAGE仍是10字节的头部+数据部分,占用字节数不变,但操作变得简单,可以直接对结构体属性赋值。
 属性类型全部改用unsigned short,表示0-65535范围的无符号整数。
     
TCP是字节流传送(序列号每次增加的值为字节数),本协议简单地使用序列号递增模式。
即SEQ每次递增1(等于接收到的ACKnum,可以对65535取模),ACKnum是对方序列号加1(表示期待对方下次发送的序列号):send\_SEQ=recv\_ACKnum,send\_ACKnum=recv\_SEQ。
本实验中SEQ主要在发送端使用,ACKnum主要在接收端使用。

        数据部分占65300字节。
        头部包括:序列号、确认序号、检验和字段、标志位字段、长度字段。每一部分占2字节,共占10字节。

        

    \begin{lstlisting}
    Header:10 bytes
    | 2 bytes  | 2 bytes  | 2 bytes  | 2 bytes  |  2 bytes  |
    |   SEQ    |  ACKnum  | CheckSum |	  ID	|   Length  |
    \end{lstlisting}

        SEQ是序列号,范围是0到65535,主要在发送端使用。

        ACKnum是确认序列号,主要在接收端使用,接收端回复的ACKnum等于发送端的SEQ+1,(代表下次期望收到的发送端序列号)。

        CheckSum是检验和字段,发送端制作分组时计算检验和填入,接收端对收到的分组进行差错检验。

        ID是多位标志位,包括SYN(连接建立标志位)、ACK(确认标志位)、FIN(断开连接标志位)。
        
    \begin{lstlisting}
    ID:2 bytes
    |  FIN  |  ACK  |  SYN  |
    \end{lstlisting}

    Length是数据长度。


    \subsection{建立连接、差错检测、确认重传、流量控制等功能}
    在附录(\ref{附录})中给出了实验3-2的基本内容,可以参考理解实验3-3。

    
    
    
    \subsection{拥塞控制(RENO算法)}

    拥塞控制是作用于网络的,它是防止过多的数据注入到网络中,避免出现网络负载过大的情况。
    本实验采用基于窗口的方法,通过拥塞窗口的增大或减小控制发送速率。

    设置三种状态:慢启动、拥塞避免、快速恢复。根据不同状态之间的转移来调整窗口大小。实验中做了一些调整,与TCP的拥塞控制有一些不同。

            \subsubsection{慢启动}
            连接建立后,进入慢启动阶段,初始拥塞窗口CWND大小为1,每当接收到一个ACK,CWND加1,因此每个RTT,CWND翻倍。
            但本实验中发送端一次发送多个分组,可能只收到最后一个分组,例如发送分组1到5,可能先收到分组5的确认,这样窗口能够加速滑动(在实验3-2中已经说明)。
            所以如果每次收到ACK只进行CWND加1的操作,不能正确达到“慢启动”的效果,窗口增加的过慢。

            因此设置为每次收到一个ACK,CWND就翻倍。


    
            \subsubsection{拥塞避免}

            设置阈值SSTHRESH,拥塞窗口达到阈值,就进入拥塞避免阶段。这里也是和TCP有一些差别。

            TCP中是对于每个RTT,CWND增加1:$CWND=CWND+MSS*MSS/CWND$,也就是对于每个ACK,CWND只增加$MSS/CWND$,这是因为TCP是基于字节计数。

            由于本实验直接使用窗口计数,且在实验3-2滑动窗口的实现中存在收到超前ACK的情况,这里设置为对于每个ACK,CWND增加1。

            所以在慢启动和拥塞避免阶段,对于收到一个ACK的结果,CWND的增长模式分别为翻倍和加1。所以本实验与TCP的区别是CWND的变化周期是以收到一个ACK为单位而不是以一个RTT为单位。


            \subsubsection{快速恢复}

            通过三次冗余ACK来检测分组丢失,执行快速重传(这也已经在3-2实现),之后进入快速恢复状态。



            \subsubsection{三种状态的转换}

            下图是reno算法的状态机。

            \begin{figure}[H] 
                \centering 
                \includegraphics[width=0.8\textwidth]{image/reno.PNG} 
                \label{png2} 
            \end{figure}

            每种状态都有一个触发条件:
            
            1.无论何种状态,当发生超时事件,就进入慢启动阶段。
           
            2.当慢启动状态的拥塞控制窗口达到阈值,或者快速恢复状态下收到新的ACK,就进入拥塞避免状态。

            3.无论何种状态,当收到三次冗余的ACK,就进入快速恢复状态。


            
           
    

    \subsection{其他设计}
        \subsubsection{定时器的设计}
        采用非阻塞模式,设置一个超时时间(设置为50ms),超过此时间无响应就返回一个值(-1),
        这样编程更方便,也实现了“定时器”的功能。经过测试,超时未响应recvfrom函数的返回值为-1,错误类型WSAGetLastError() 的返回值为 10060。
    \begin{lstlisting}
    // 设置发送超时
    setsockopt(socket_client,SOL_SOCKET,SO_SNDTIMEO, (char *)&nNetTimeout,sizeof(int));
    //设置接收超时
    setsockopt(socket_client,SOL_SOCKET,SO_RCVTIMEO, (char *)&nNetTimeout,sizeof(int));
    \end{lstlisting}
        
        \subsubsection{随机丢包和延时的设计}
        使用随机数取模的方法,随机产生丢包,包括发送端丢弃数据包、丢弃重传的数据包,接收端丢弃ACK等等。使用Sleep函数进行延时。

        \subsubsection{快速重传和窗口加速滑动}

        实现了快速重传,发送端维护一个变量,代表期望接收的ACK序号,如果发送端收到了三次相同的冗余ACK,就执行快速重传,立即重传当前滑动窗口内的所有分组。
        如果收到了超前的ACKnum,比如发送了分组1、2、3,期待收到ACKnum=2、3、4,如果先收到的是4,,就可以加速滑动窗口,而不是每次滑动一位。
        这种情况在实际运行中是会发生的。

        \subsubsection{测试方法}
        将输入输出重定向到文件,实现运行日志的保存。在windows环境下使用批处理文件实现快捷测试。
    \begin{lstlisting}
    批处理命令:
    test_server < input_server >> output_server & exit
    test_client < input_client >> output_client & exit
    python文件作用是将输入参数(窗口大小、文件序号)写入文件并运行批处理命令。
    运行以下命令或将此命令写入bat文件运行,即可传输文件1:
    python s.py 1 
    \end{lstlisting}



    \section{代码分析——在发送端进行三种状态的转换}

    实际编程中只需要在3-2的基础上增加三种状态的转换即可。

    定义三种状态,并在开始时初始化为慢启动状态。

    \begin{lstlisting}
    #define SLOW 1
    #define AVOID 2
    #define QUICK 3
    \end{lstlisting}

    慢启动状态。
    \begin{lstlisting}
    if (recv_Ret < 0)
    {
        state=SLOW;
        SSTHRESH/=2;
        CWND=1;
        cout<<"进入慢启动 CWND: "<<CWND<<endl;
        ...
    }
    \end{lstlisting}
 
 
    拥塞避免状态。
    \begin{lstlisting}
     if(CWND>=SSTHRESH)
     {
        state=AVOID;
        cout<<"进入拥塞避免 CWND:"<<CWND<<endl;
     }
    \end{lstlisting}
    
    快速恢复状态。
    \begin{lstlisting}
     if(dup_ack_num==2)
     {
         dup_ack_num=0;
         state=QUICK;
         SSTHRESH=CWND/2;
         CWND=SSTHRESH+3;
         cout<<"进入快速恢复 CWND: "<<CWND<<endl;
         
         goto LABEL_TRANS_AGAIN;//进行重传
         
     }
    \end{lstlisting}
 
 
    收到ACK的动作。
    \begin{lstlisting}
     if(state==SLOW)
     {
        CWND*=2;
     }
     else if(state==AVOID)
     {
        CWND+=1;
     }
     else
     {
         state=AVOID;
         CWND=SSTHRESH;
         cout<<"进入拥塞避免 CWND: "<<CWND<<endl;
         
     }
    \end{lstlisting}
    
 
             
         
 
         
 \section{结果展示}
 连接建立后,拥塞控制窗口从1开始,先进入慢启动阶段,每收到一个ACK,CWND翻倍,之后CWND达到阈值,进入拥塞避免阶段,当收到三次冗余ACK还会进行快速恢复。
 
 下面是程序运行日志。
 \begin{lstlisting}
     connect!
     请输入要传输的文件序号(1 2 3 4 5):4
     ***************start transport****************
     ~~~~~~~~~~~~CWND:1~~~~~~~~~~~~~~~
     send seq: 1
     successfully trans 1   CWND+=1 !!!
     ~~~~~~~~~~~~CWND:2~~~~~~~~~~~~~~~
     send seq: 2
     send seq: 3
     successfully trans 2   CWND+=1 !!!
     ~~~~~~~~~~~~CWND:4~~~~~~~~~~~~~~~
     send seq: 4
     send seq: 5
     send seq: 6
     ----------收到超前的ACKnum,可以加速滑动!!! ACKnum: 5 CWND+=1!!!
     ~~~~~~~~~~~~CWND:8~~~~~~~~~~~~~~~
     send seq: 7
     send seq: 8
     send seq: 9
     send seq: 10
     send seq: 11
     *********随机丢弃数据包 12*********
     ----------收到超前的ACKnum,可以加速滑动!!! ACKnum: 8 CWND+=1!!!
     ~~~~~~~~~~~~CWND:16~~~~~~~~~~~~~~~
     ######进入拥塞避免 CWND:16 ######
     send seq: 13
     send seq: 14
     *********随机丢弃数据包 15*********
     send seq: 16
     send seq: 17
     *********随机丢弃数据包 18*********
     send seq: 19
     *********随机丢弃数据包 20*********
     send seq: 21
     send seq: 22
     send seq: 23
     ----------收到超前的ACKnum,可以加速滑动!!! ACKnum: 12 CWND+=1!!!
     ~~~~~~~~~~~~CWND:17~~~~~~~~~~~~~~~
     ######进入拥塞避免 CWND:17 ######
     send seq: 24
     send seq: 25
     send seq: 26
     收到之前的ACKnum: 12!不执行动作即可
     ~~~~~~~~~~~~CWND:17~~~~~~~~~~~~~~~
     ######进入拥塞避免 CWND:17 ######
     ######进入慢启动 CWND: 17 ######
     超时未接收到ACK,重传 27-12 == 15个数据包!!!
     ----------收到超前的ACKnum,可以加速滑动!!! ACKnum: 14 CWND+=1!!!
     ~~~~~~~~~~~~CWND:2~~~~~~~~~~~~~~~
     ######进入慢启动 CWND: 2 ######
     超时未接收到ACK,重传 27-14 == 13个数据包!!!
     ######进入慢启动 CWND: 1 ######
     超时未接收到ACK,重传 27-14 == 13个数据包!!!
     ######进入慢启动 CWND: 1 ######
     超时未接收到ACK,重传 27-14 == 13个数据包!!!
     ######进入慢启动 CWND: 1 ######
     文件helloworld.txt(1655808 bytes)传输完成!
 
     传输用时:5.275秒
     传输数据量(不包含重复数据):1655808 bytes ( 1617KB )
     吞吐率:313897 bytes/s ( 306.54KB/s )
 
     **************完成一次测试!!!**************
 \end{lstlisting}
    
    
    
    

    



\section{总结反思}
通过编程实现更熟悉了流量控制和拥塞控制的区别。
流量控制是作用于接收者的,它是控制发送者的发送速度从而使接收者来得及接收,根本目的是防止分组丢失,
它是构成TCP可靠性的一方面。
拥塞控制是作用于网络的,是防止过多的数据注入到网络中,避免出现网络负载过大的情况。










\section{附录1:实验3-2的功能实现\label{附录}}
\subsection{建立与断开连接}
假设a为客户端(发送)b为服务端(接收),下面是交互过程的简单描述。

SYN和FIN只在开始和结束时使用。

\subsubsection{建立连接}
假设初始X=0,Y=0,这里的序列号从0开始,也可使用随机数取模产生随机序列号。
\begin{lstlisting}
a->b:SYN=1  ACK=0  SEQ=X=0  ACKnum=0
b->a:SYN=1  ACK=1  SEQ=Y=0  ACKnum=X+1=1
a->b:SYN=0  ACK=1  SEQ=1    ACKnum=1   第三次握手的同时发送数据
\end{lstlisting}
        
        
\subsubsection{断开连接}
假设初始X=1,Y=1,模仿TCP四次挥手(https://www.jianshu.com/p/cd801d1b3147)但这里三次交互就够,
a不需要第二次等待,b收到FIN后立即发送FIN。
\begin{lstlisting}
a->b:FIN=1  ACK=1  SEQ=X=1  ACKnum=1
b->a:FIN=1  ACK=1  SEQ=Y=1  ACKnum=2
a->b:FIN=0  ACK=1  SEQ=2   ACKnum=2
\end{lstlisting}
        
        断开连接时,a先发送FIN信号。b接收到FIN之后,也发送FIN(让对方知道自己知道了要断开连接)。
        a发送FIN,再收到FIN之后,发送ACK(这里对方收不到可以重传),然后退出。
        b发送FIN并且收到ACK之后退出。

\subsection{差错检测}
采用UDP校验和计算方法。发送端在发送分组时进行UDP校验和计算,结果写入CheckSum位。接收端接收到分组后也进行校验和计算,结果为0xFFFF则无错误。

计算校验和时,本实验没有添加伪首部,只是对头部和数据部分计算。首先需要把char型的字符数组(char类型占1字节)转为unsigned short型的数组(unsigned short类型占2字节),
采用反码求和方法(加法的溢出位需要回卷),需要注意头部占10字节,数据部分的字节数若为奇数,需要对最后一个字节进行单独处理(可以直接拷贝到unnsiged short类型的数组,这与C++在x86下属于小端编址有关)。
\subsection{确认重传}
在上次实验的停等协议中,发送端收到确认序号ACKnum后再次发送下一个分组,SEQ=ACKnum。接收端进行累积确认。
如果发送端发送的分组丢失,接收端一直等待接收不发送ACK,或者接收端发送的ACK丢失,都会导致发送端收不到ACKnum,
这时发送端就重传窗口内还未确认的所有分组。
 
\subsection{基于滑动窗口的流量控制机制}
本实验采用GBN方法,大小窗口固定为20,注意GBN方法的窗口大小需要小于序列号数量即$GBN<=2^n-1$,SR方法窗口大小需要满足$SR<=2^{n-1}$。

滑动窗口的思想可以简单理解为,发送端一次发送多个分组。接收端依旧使用累积确认,按序接收,当收到失序的分组时直接丢弃,并发送原来的ACKnum(期待按序收到的分组序号)。
发送端收到ACK后,滑动窗口。若发生超时事件或收到重复的与之前相同的ACK,代表接收端没有收到具有ACKnum序号的分组,需要进行重传。




        \subsubsection{发送端}

        发送端需要响应三种类型的事件:

        1.上层调用。本实验使用的是同步机制,仅当窗口不满时才读取文件进行打包传输。

        2.收到一个ACK(ACKnum为n)。代表接收端已经接收分组n以及之前的分组,这就是累积确认,这一点很重要,方便之后发送端窗口的加速滑动。

        3.超时事件。本实验使用的是非阻塞方式,通过设置发送和接收的超时时间,也能很好实现“定时器”的功能,并且似乎能产生加速的效果。按照GBN的思想,例如发送端发送多个分组,序号从1到10,分组10发送最晚,接收端最晚接收,假设只有分组10损坏,只有等接收端发送完毕ACK9之后发送端才能知道第10个分组损坏,需要重传。
        但在编程中发现,在本实验中不使用定时器(一个原因包括可能需要涉及线程较为复杂)的非阻塞方式中,发送端可能在发送完这10个分组后,先收到分组9的ACK而不是分组1的ACK(注意因为累积确认的原则,接收端肯定发送了ACK2到ACK10),
        但发送端先收到ACK10,这窗口就可以加速滑动,之前的ACK就可以忽略了。
            
        发送端要负责维护一个发送分组的缓存便于之后可能的重传。
        
        可以使用队列,队列中的元素代表着当前窗口——已发送但还未确认的分组。发送的分组都加入队列,发送后经确认的分组可以从队列移出。重传时只需重传队列中的所有分组。
       
        下图是发送端的扩展状态机:

\begin{figure}[H] 
    \centering 
    \includegraphics[width=0.6\textwidth]{image/GBN.PNG} 
    \label{png2} 
\end{figure}
       
        \subsubsection{接收端}

        接收端的处理很简单。如果收到按序的分组n(即上次收到的分组序号是n-1),则接收并回复ACKnum=n+1,
        接收到失序的分组,直接丢弃,并发送原来的ACKnum(期待按序接收的分组序号)。




% -------------------------------参考文献------------------------------------





\section{附录2:实验3-2的代码分析}
    \subsection{公共头文件定义}
    \begin{lstlisting}
    //设置IP和端口号等
    #define server_Port 1001
    #define server_IP "127.0.0.1"
    #define client_Port 1002
    #define client_IP "127.0.0.1"
    //rand()%RAND_MOD_NUM 对随机数取模来决定发送还是丢弃分组
    int RAND_MOD_NUM=15;
    //非阻塞模式的超时时间
    int nNetTimeout=50;//毫秒
    //缓冲区大小
    #define BUF_LEN 65310//比2**16小一点
    #define HEADER_LEN 10//头部长度
    #define DATA_LEN 65300
    typedef unsigned short ushort;
    /*  
    头部的设计
    Header:10 bytes
    | 2 bytes  | 2 bytes  | 2 bytes  | 2 bytes  |  2 bytes  |
    |   SEQ    |  ACKnum  | CheckSum |	  ID	|   Length  |
    
    ID:2 bytes
    |   FIN    |  ACK  |  SYN  |
    */
    struct PACKAGE
    {
        ushort SEQ;//序列号 2bytes,16bits 0-65535
        ushort ACKnum;//确认序号
        ushort CheckSum;//检验和
        ushort ID;//多个标志位
        ushort Length;//数据长度,这是基本固定的
        char data[DATA_LEN];
    };
    //各个标志位
    #define SYN 0x1//建立连接
    #define ACK 0x2//确认
    #define FIN 0x4//断开连接
    \end{lstlisting}

    \subsection{基本函数(不传入参数,直接执行完成相应功能)}
            \subsubsection{makePackage}
                    用于制作分组的头部,发送端需要设置校验和、发送序列号、数据长度。接收端需要设置确认序号。
    \begin{lstlisting}
    sendbuf.Length = length;
    sendbuf.SEQ =(sendbuf.SEQ+1)%65536;
    sendbuf.CheckSum = 0;
    memcpy(&sendbuf, &header, HEADER_LEN);
    \end{lstlisting}
                    注意建立连接时填充SYN字段,结束连接时填充FIN字段。发送端在读取数据后,需要再次填充CheckSum字段。
            \subsubsection{doCheckSum}
                    计算UDP校验和的函数。发送端计算校验和填入CheckSum字段。接收端计算校验和判断是否等于0xFFFF,不等于0xFFFF则数据出错。
    \begin{lstlisting}
    //先将char型发送缓冲区转换成为16位的unsigned short型数据
    //需要分情况,考虑缓冲区长度
    array = new unsigned short[count];
    for (int i = 0; i < count; i++)
    {
        memcpy(&array[i], &sendbuf[i * 2], 2);
    }
    //计算校验和
    while (count--)
    {
        sum += array[i++];
        sum = (sum >> 16) + (sum & 0xFFFF);
    }
    sendbuf.CheckSum=~sum;
    \end{lstlisting}
            \subsubsection{Check\_expected\_ACK(发送端使用)}
                    发送端检查接收的ACKnum是否是期望且按序的(即等于最早未确认的分组序号加1)。如果等于期望的ACKnum,窗口滑动一位。如果小于期望的ACKnum,直接忽略即可。如果大于期望的ACKnum,则窗口可以加速滑动。
                    每次收到一个按序的ACKnum后,expected\_ack进行更新:$expected\_ack=recvbuf.ACKnum+1;$。

            \subsubsection{CheckSEQ\_EXPECTED(接收端使用)}
            接收端检查接收到的分组是否按序,比较接收到的分组序号是否等于刚刚发送的ACKnum即可。

            
            \subsubsection{IsACK,IsFIN,IsSYN}
                    检查这些标志位是否置位。
    \begin{lstlisting}
    bool IsFIN()
    {
        if ((recvbuf.ID & FIN) != 0) 
            return true;
        else
            return false;
    }
    \end{lstlisting}
    

    \subsection{发送端代码}
            先是WinSocket的初始化,创建socket,设置超时时间等操作。
           
            第一步发送包含SYN的分组,准备建立连接。
            连接建立后,先发送文件名。之后开始读文件(每次读取数据缓冲区大小的数据量)。制作分组(包括序列号每次取模递增,数据长度,待装入数据后计算填充校验和)。
            发送分组,可能会随机丢弃。
            
            发送端维护一个队列Send\_Buf\_Queue用于缓存发送的分组。
            滑动窗口具有p\_Base、p\_NextSeqNum两个指针,如下图所示,分别指向最早的发送未确认分组和即将发送的分组。
            在发送分组时,将分组加入发送缓冲队列,在接收到ACK后,将队头元素(已确认的分组)出队列。

            \begin{figure}[H] 
                \centering 
                \includegraphics[width=0.8\textwidth]{image/basenext.PNG} 
                \label{png3} 
            \end{figure}


            GBN的交互主要包括窗口滑动、接收ACk、超时重传三种情况。
            \begin{figure}[H] 
                \centering 
                \includegraphics[width=0.8\textwidth]{image/GBN_1.PNG} 
                \label{png3} 
            \end{figure}
        \subsubsection{窗口滑动}
        初始时,base和nextseqnum都为1,先发送多个分组。
        收到期待的ACKnum后,base加1,收到超前的ACKnum后,base增加值大于1。
        base变化后,p\_NextSeqNum<p\_Base+WINDOW\_WIDTH条件就满足了,进入while循环,窗口实现滑动。

    \begin{lstlisting}
    /*****************读取、发送*****************
    * 0 1 2 3 4 5 6 7 8 9
    *   |         |    n=5(1 2 3 4 5)
    *   base     next
    * */
    while((p_NextSeqNum<p_Base+WINDOW_WIDTH)&&read_flag)//这里是while!!!不是if!!!!
    {
        if(p_NextSeqNum==SEQ_SIZE)
        {//实现循环,这样也可以很简单
            p_Base-=p_Base;p_NextSeqNum-=p_Base;
        }
        file.read((char*)&sendbuf.data, DATA_LEN);
        if (file.gcount() < DATA_LEN)
        {
            //读取结束了
            sendbuf.ID |= FIN;//或操作使得FIN位置为1
            read_flag=0;//不再读取了
        }
        makePackage(file.gcount());
        //设置校验和并填充
        doCheckSum(file.gcount());
        //加入发送缓冲队列
        PACKAGE t1=sendbuf;
        Send_Buf_Queue.push(t1);
        //发送数据包,随机丢弃
        if(rand()%RAND_MOD_NUM!=0)//即遇到rand()%RAND_MOD_NUM==0丢弃
        {
            //宏替换语句,包括发送和延时sendto(socket_client, (char*)&sendbuf, BUF_LEN, 0, (SOCKADDR*)&server_addr, sizeof(SOCKADDR));Sleep(SEND_TIME_DELAY)
            CLIENT_SEND;
            cout<<"send seq: "<<sendbuf.SEQ<<endl;//cout<<"发送第 "<<sendbuf.SEQ<<" 个数据包!!!"<<endl;
        }
        else
        {
            cout<<"*********随机丢弃数据包 "<<sendbuf.SEQ<<"*********"<<endl;
        }
        
        p_NextSeqNum++;
        
    }
    \end{lstlisting}


        \subsubsection{接收ACK}
            这里根据收到的ACKnum和期待的ACKnum不同,又分为三种情况。

            1.recvbuf.ACKnum==expected\_ack

            直接移动base,发送端缓存队列队头元素出队。

    \begin{lstlisting}
    cout << "successfully trans " << recvbuf.ACKnum-1 << endl;
    trans_bytes+=recvbuf.Length;
    p_Base=recvbuf.ACKnum;
    Send_Buf_Queue.pop();
    \end{lstlisting}

            2.recvbuf.ACKnum<expected\_ack

            记录一个当前收到的ACKnum,当连续三次收到冗余的ACK时,就执行快速重传。
    \begin{lstlisting}
    if(recvbuf.ACKnum==cur_ack)
    {
        dup_ack_num++;
        cout << "收到之前的ACKnum且重复!!!: " << recvbuf.ACKnum << ""<< endl;
    }
    else
    {
        dup_ack_num=0;
        cout << "收到之前的ACKnum: " << recvbuf.ACKnum << "!不执行动作即可"<< endl;
        cur_ack=recvbuf.ACKnum;
    }
    if(dup_ack_num==2)
    {
        //快速重传
        dup_ack_num=0;
        cout<<"********************执行快速重传!!!*******************"<<endl;
        goto LABEL_TRANS_AGAIN;
    }
    \end{lstlisting}

            3.recvbuf.ACKnum>expected\_ack

            收到超前的ACKnum,base多移动几位,窗口能加速滑动。

    \begin{lstlisting}
    cout<<"收到超前的ACKnum,可以加速滑动!!! ACKnum: "<<recvbuf.ACKnum<<endl;
    p_Base=recvbuf.ACKnum;
    //p_Base移动
    for (int i = 0; i < recvbuf.ACKnum-expected_ack+1; i++)
    {
        Send_Buf_Queue.pop();
    }
    expected_ack=recvbuf.ACKnum+1;
    trans_bytes+=(recvbuf.ACKnum-expected_ack+1)*DATA_LEN;
    \end{lstlisting}

        \subsubsection{超时重传}
        
        超时未收到ACK,重传窗口内的所有分组,只需要把发送缓冲队列里的分组依次出队列、发送、再加入队列。
        
    \begin{lstlisting}				
    /* base--next-1
    * 0 1 2 3 4 5 6 7 8
    *     |       | 已发送1 2 3 4 5
    * 收到ack2,收不到ack3,可能情况:1.对方未收到seq3,期待收到seq3 2.对方收到了seq3,但是ack3丢失
    * 都执行重传,重传2 3 4 5 6
    */
    for (int i = p_Base; i < p_NextSeqNum; i++)
    {
        sendbuf=Send_Buf_Queue.front();
        if(rand()%RAND_MOD_NUM!=0)
        {
            CLIENT_SEND;
        }
        else
        {
            cout<<"*********************随机丢弃重传的数据包!!!!!******************"<<endl;
        }
        PACKAGE t1=Send_Buf_Queue.front();
        Send_Buf_Queue.push(t1);
        Send_Buf_Queue.pop();
    }
    cout << "超时未接收到ACK,重传 "<<p_NextSeqNum<<"-"<<p_Base<<" == "<<p_NextSeqNum-p_Base<<"个数据包!!!" << endl;
    \end{lstlisting}
            

    \subsection{接收端代码}
            和发送端相同,建立连接后接收文件名。之后开始接收分组,写入文件,发送ACK确认分组(也进行随机丢弃)。
        
            CheckSum函数进行校验和计算,CheckSEQ\_EXPECTED确保接收到的是按序期望的不重复的分组,否则丢弃即可,IsFIN判断是否传输完成。
        
        \subsubsection{接收按序的分组并写入文件}
    \begin{lstlisting}
    cout << "successfully recv " << recvbuf.SEQ << endl;
    trans_bytes += recvbuf.Length;
    file.write(recvbuf.data, recvbuf.Length);
    makePackage();
    if(rand()%RAND_MOD_NUM!=0)
    {
        SERVER_SEND;
    }
    else
    {
        cout<<"**********************随机丢弃ACK包!!!!!**********************"<<endl;
    }
    \end{lstlisting}

        \subsubsection{接收到失序的分组,直接丢弃}
    \begin{lstlisting}
    /*******************************收到失序(超前)的包************************************************
    * 比如收到了seq5,回复acknum==6,期望的是seq6
    * 结果收到seq8,直接丢弃
    * package.ACKnum不变
    */
    cout<<"接收到失序的数据包 "<<recvbuf.SEQ<<" ,丢弃,发送原ACKnum: "<<sendbuf.ACKnum<<"!!!"<<endl;
    if(rand()%RAND_MOD_NUM!=0)//即遇到rand()%RAND_MOD_NUM==0丢弃
    {
        SERVER_SEND;
    }
    else
    {
        cout<<"**********************随机丢弃ACK包(收到失序数据包时的ACK)!!!!!**********************"<<endl;
    }
    \end{lstlisting}


        \subsection{断开连接的其他辅助处理}
        
        \subsubsection{发送端可能收不到接收端回复的FIN}
        发送端在发送最后一个分组的同时发送FIN,但有可能收不到接收端的FIN,这时base和nextseqnum已经相等,可以根据此来决定终止连接。

    \begin{lstlisting}
    if((p_Base==p_NextSeqNum&&p_Base!=1))
    {
        cout<<"已经结束了!!!"<<endl;
        goto LABEL_Complete;
    }	
    \end{lstlisting}

        另外也可能遇到接收端退出过快,发送端甚至收不到ACKnum。注意接收端只有接收完毕并且收到FIN和ACK才会退出。
        可以判断发送端重传次数,如果多次重传相同的分组,则有可能是接收端已经退出了,发送端退出即可。
    \begin{lstlisting}
    if(p_Base==T_a&&p_NextSeqNum==T_b)
    {
        dup_trans_num++;
    }
    else
    {
        dup_trans_num=0;
        T_a=p_Base;T_b=p_NextSeqNum;
    }
    if(dup_trans_num==20)
    {
        cout<<"对方无应答,应该是退出了,结束!!"<<endl;
        goto LABEL_Complete;
    }
    \end{lstlisting}

        对于接收端,必须收到发送端的FIN和ACK才能退出。

    \begin{lstlisting}
    /*发送端发来了FIN,接收端再发一次FIN,等到发送端知道了,发送ACK(发送端只有这时候发送ACK),之后再退出*/
    while(recvbuf.ID&ACK==0)
    {
        cout<<"send FIN!"<<" ";
        SERVER_SEND;//含有FIN
        do
        {
            recv_Ret = recvfrom(socket_server, (char*)&recvbuf, BUF_LEN, 0, (SOCKADDR *)&client_addr, &recv_para_len);
        } while (recv_Ret < 0);
    }
    \end{lstlisting}

        

        
\section{附录3:实验3-2的结果展示}
程序可以连续发送任意类型的文件,也可以选择输入窗口大小、文件名等等,能够实现文件的无损正确传输。
超时时间越短(这里是10ms),缓冲区越大(这里是比2的16次方小一些,不能超过,取65310),文件发送速度越快。

以下是窗口大小为20的测试结果。

    从下图可以看出,发生过发送端丢弃分组和接收端丢弃ACK的情况,也发生了窗口加速滑动。
    
    \begin{figure}[H] 
        \centering 
        \includegraphics[width=0.8\textwidth]{image/1_trans_again.PNG} 
        \label{png2} 
    \end{figure}

    
    下图是超时重传,可以看到第一次重传从分组15开始,之后分组15得到确认。但发送端随即丢弃了分组16,导致第二次重传。
    这时也发生了接收端退出过早的情况,可以通过检查发送端是否多次重传相同分组来确认。
    \begin{figure}[H] 
        \centering 
        \includegraphics[width=0.8\textwidth]{image/2_quit.PNG} 
        \label{png2} 
    \end{figure}

    下图也包含了超时重传,发送端的三次重传分别从分组13、14、25开始。
    \begin{figure}[H] 
        \centering 
        \includegraphics[width=0.8\textwidth]{image/3.PNG} 
        \label{png2} 
    \end{figure}


    \end{document}




            
  
